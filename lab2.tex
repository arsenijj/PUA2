\documentclass[bachelor, och, labwork]{shiza}

\usepackage{subfigure}
\usepackage{tikz,pgfplots}
\pgfplotsset{compat=1.5}
\usepackage{float}
\usepackage{pdfpages}

\usepackage{titlesec}
\setcounter{secnumdepth}{4}
\titleformat{\paragraph}
{\normalfont\normalsize}{\theparagraph}{1em}{}
\titlespacing*{\paragraph}
{35.5pt}{3.25ex plus 1ex minus .2ex}{1.5ex plus .2ex}

\titleformat{\paragraph}[block]
{\hspace{1.25cm}\normalfont}
{\theparagraph}{1ex}{}
\titlespacing{\paragraph}
{0cm}{2ex plus 1ex minus .2ex}{.4ex plus.2ex}

% --------------------------------------------------------------------------%


\usepackage[T2A]{fontenc}
\usepackage[utf8]{inputenc}
\usepackage{graphicx}
\graphicspath{ {./images/} }
\usepackage{tempora}

\usepackage[sort,compress]{cite}
\usepackage{amsmath}
\usepackage{amssymb}
\usepackage{amsthm}
\usepackage{fancyvrb}
\usepackage{listings}
\usepackage{listingsutf8}
\usepackage{longtable}
\usepackage{array}
\usepackage[english,russian]{babel}

\usepackage[hidelinks]{hyperref}
\usepackage{url}

\usepackage{underscore}
\usepackage{setspace}
\usepackage{indentfirst} 
\usepackage{mathtools}
\usepackage{amsfonts}
\usepackage{enumitem}
\usepackage{tikz}
\usepackage{minted}

\newcommand{\eqdef}{\stackrel {\rm def}{=}}
\newcommand{\specialcell}[2][c]{%
\begin{tabular}[#1]{@{}c@{}}#2\end{tabular}}

\renewcommand\theFancyVerbLine{\small\arabic{FancyVerbLine}}


\begin{document}
\includepdf{titul.pdf}

%-------------------------------------------------------------------------------

\tableofcontents

\intro

Бинарные отношения могут быть эквивалентными, и, поэтому на них могут строиться
фактор-множества. Если же бинарное отношение не является эквивалентностью, то
по определенному алгоритму можно построить эквивалентное замыкание данного
отношения. Также отношения могут обладать определенным порядком, в зависимости
от конкретных свойств. Если же отношение обладает порядком, то для данного
отношения можно построить диаграмму Хассе, а также для него могут быть
найдены минимальные и максимальные, и наименьшие и наибольшие элементы.
Также для бинарных отношений определены понятия контекста и концепта, а также
существует алгоритм вычисления решетки концептов.

\section{\textbf{Цель работы и порядок ее выполнения}}

\textbf{Цель работы} "--- изучение основных свойств бинарных отношений и 
операций замыкания бинарных отношений.

Порядок выполнения работы:

\begin{enumerate}

    \item Разобрать определения отношения эквивалентности, фактор-множества. 
    Разработать алгоритмы построения эквивалентного замыкания бинарного отношения 
    и системы представителей фактор-множества.  

    \item Разобрать определения отношения порядка и диаграммы Хассе. Разработать 
    алгоритмы вычисления минимальных (максимальных) и наименьших (наибольших) 
    элементов  и построения диаграммы Хассе. 

    \item Разобрать определения контекста и концепта. Разработать алгоритм 
    вычисления решетки концептов.

\end{enumerate}

\section{Теоретические сведения}

% \subsection{Основные определения видов бинарных отношений и их алгоритмы}

% \subsubsection{Определение бинарного отношения}

% Подмножества декартова произведения $A \times B$ множеств $A$ и $B$ называются
% \textbf{бинарными отношениями} между элементами множеств $A$ и $B$ и 
% обозначаются строчными греческими буквами $\rho$, $\rho_1$ и т.п.

% Для бинарного отношения $\rho\subset A \times B$ область определения $D_\rho$ и 
% множество значений $E_\rho$ определяется как подмножества соответствущих множеств 
% $A ~\text{и}~ B$ по следующим формулам:

% \begin{center}
%     $D_\rho ~ = ~ \{a : (a, b) \in\rho ~ \text{для некоторого} ~ b \in B\}$,
%     $E_\rho ~ = ~ \{b : (a, b) \in\rho ~ \text{для некоторого} ~ a \in A\}$.
% \end{center}

% \subsubsection{Основные свойства бинарных и алгоритмы их определения}

% Бинарное отношение $\rho \subset A \times A$ называется:

% \begin{enumerate}
    
%     \item \textit{рефлексивным}, если $(a,a)\in\rho~\forall a \in A$.
    
%     \item \textit{антирефлексивным}, если $(a,a)\not\in\rho~\forall a \in A$.
    
%     \item \textit{симметричным}, если $(a,b)\in\rho\Rightarrow (b,a)\in\rho~\forall a,b\in A$.
    
%     \item \textit{антисимметричным}, если $(a,b) \in\rho~\text{и}~(b,a) \in\rho ~ \Rightarrow a = b ~\forall a,b \in A$.
    
%     \item \textit{транзитивным}, если $(a,b)\in\rho ~\text{и}~ (b,c)\in\rho\Rightarrow (a,c) \in\rho ~\forall a,b,c \in A$.

% \end{enumerate}

% Далее представлена программная реализация определения свойств бинарных 
% отношений.

% Пусть $\rho$ - бинарное отношение на множестве $A=\{a_1,...,a_N\}$ мощности $N$.
% Тогда \textit{матрицей} бинарного отношения $\rho$ будет матрица $M(\rho)$ 
% размерности $N \times N$, определяемая следующим образом: 


%     \begin{equation*}
%         \forall i,j = \overline{1,N} \quad M(\rho)_{ij} = 
%             \begin{cases}
%                 1, &\text{если $(a_i,a_j) \in \rho$}\\
%                 0, &\text{если  $(a_i,a_j) \not\in \rho$}
%             \end{cases}
%     \end{equation*}


% Выполним проверку свойств \textbf{рефлексивности} и \textbf{антирефлексивности}:

% Алгоритм 1. Проверка бинарного отношения на \textit{рефлексивность}.

% \textit{Вход.} Матрица $M(\rho)$ бинарного отношения $\rho$ размерности
% $N \times N$.

% \textit{Выход.} <<Отношение является рефлексивным>> или "Отношение не является 
% рефлексивным".

% \begin{enumerate}
        
%     \item Цикл по $i ~\text{от}~ 1 ~\text{до}~ N$.

%     \item Если $M_{ii} = 0$, то ответ <<Отношение не является рефлексивным>>. 

%     \item Если цикл завершен то ответ <<Отношение является рефлексивным>>.
    
% \end{enumerate}
% Трудоемкость алгоритма $O(n)$.

% Алгоритм 2. Проверка бинарного отношения на \textit{антирефлексивность}.

% \textit{Вход.} Матрица $M(\rho)$ бинарного отношения $\rho$ размерности
% $N \times N$.

% \textit{Выход.} <<Отношение является антирефлексивным>> или <<Отношение не является 
% антирефлексивным>>.

% \begin{enumerate}
        
%     \item Цикл по $i ~\text{от}~ 1 ~\text{до}~ N$.

%     \item Если $M_{ii} = 1$, то ответ "Отношение не является антирефлексивным. 

%     \item Если цикл завершен то ответ "Отношение является антирефлексивным.
    
% \end{enumerate}
% Трудоемкость алгоритма $O(n)$.

% Выполним проверку свойств \textbf{симметричности и антисимметричности}:

% Свойство симметричности выполняется для отношения, заданного матрицей, если
% элементы, симметричные относительно главной диагонали равны.

% Алгоритм 3. Проверка бинарного отношения на \textit{симметричность}.

% \textit{Вход.} Матрица $M(\rho)$ бинарного отношения $\rho$ размерности
% $N \times N$.

% \textit{Выход.} <<Отношение является симметричным>> или <<Отношение не является
% симметричным>>.

% \begin{enumerate}
%     \item Цикл по $i ~\text{от}~ 1 ~\text{до}~ N$, 
%     цикл по $j ~\text{от}~ 1 ~\text{до}~ N$.

%     \item Если $M_{ij} \not= M_{ji}$, то ответ <<Отношение не является симметричным>>.
    
%     \item Если циклы завершены, то ответ <<Отношение является симметричным>>.

% \end{enumerate}
% Трудоемкость алгоритма $O(N^2)$.

% Алгоритм 4. Проверка бинарного отношения на \textit{антисимметричность}.

% \textit{Вход.} Матрица $M(\rho)$ бинарного отношения $\rho$ размерности
% $N \times N$.

% \textit{Выход.} <<Отношение является антисимметричным>> или <<Отношение не является
% антисимметричным>>.

% \begin{enumerate}
%     \item Цикл по $i ~\text{от}~ 1 ~\text{до}~ N$, 
%     цикл по $j ~\text{от}~ 1 ~\text{до}~ N$.

%     \item Если $M_{ij} = M_{ji} ~\text{и}~ j \ne i$, то ответ <<Отношение не является антисимметричным>>.
    
%     \item Если циклы завершены, то ответ <<Отношение является антисимметричным>>.

% \end{enumerate}
% Трудоемкость алгоритма $O(N^2)$.

% Выполним проверку свойства \textbf{транзитивности}:

% Свойство транзитивности выполняется для отношения, заданного матрицей, если для 
% любого фиксированного элемента $M_{k,i}=1$ из матрицы отношения, и для любого
% элемента из матрицы отношения $M_{i,j}=1$, то выполняется $M_{k,j}=1$.

% Алгоритм 5. Проверка бинарного отношения на \textit{транзитивность}.

% \textit{Вход.} Матрица $M(\rho)$ бинарного отношения $\rho$ размерности
% $N \times N$.

% \textit{Выход.} <<Отношение является транзитивным>> или <<Отношение не является
% транзитивным>>.

% \begin{enumerate}
    
%     \item Цикл по $k ~\text{от}~ 1 ~\text{до}~ N$, 
%     цикл по $i ~\text{от}~ 1 ~\text{до}~ N$, цикл по $j ~\text{от}~ 1 ~\text{до}~ N$.
    
%     \item Если $M_{k,i}=M_{i,j}=1 ~\text{и}~ M_{k,j}=0$, то ответ <<Отношение не является
%     транзитивным>>.
   
%     \item Если циклы завершены, то ответ <<Отношение не является транзитивным>>

% \end{enumerate}
% Трудоемкость алгоритма $O(N^3)$.

% \subsection{Классификация бинарных отношений}

% Таким образом, в зависимости от свойств, которыми заданное бинарное отношение
% обладает, его можно отнести к определенному классу: \textbf{квазипорядка,
% эквивалентности} или \textbf{частичного порядка}. 

% \subsubsection{Определения классов бинарных отношений}

% Отношение \textit{эквивалентности} -- это такое бинарное отношение между элементами
% множества $A$, для которого выполнены свойства рефлексивности, симметричности 
% и транзитивности.

% Отношение \textit{квазипорядка} -- это такое бинарное отношение, между элементами
% множества $A$, для которого выполнены свойства рефлексивности и транзитивности.

% Отношение \textit{частичного порядка} -- это такое бинарное отношение, между
% элементами множества $A$, для которого выполнены свойства рефлексивности, 
% транзитивности и антисимметричности.

% \subsubsection{Алгоритм проверки отношения на квазипорядок}

% \textit{Вход.} Матрица $M(\rho)$ бинарного отношения $\rho$ размерности
% $N \times N$.

% \textit{Выход.} <<Отношение является отношением квазипорядка>> или <<Отношение 
% не является отношением квазипорядка>>.

% \begin{enumerate}

%     \item Запустить проверку свойств рефлексивности и транзитивности и выполнить
%     логическую операцию \& для их результатов.

%     \item Если значение \textit{Истина}, то ответ <<Отношение является отношением
%     квазипорядка>>.
    
%     \item Если значение \textit{Ложь}, то ответ <<Отношение не является 
%     отношением квазипорядка>>.

% \end{enumerate}
% Трудоемкость алгоритма $O(N+N^3) = O(N^3)$ в силу запуска алгоритмов проверки
% свойств рефлексивности и транзитивности.

% \subsubsection{Алгоритм проверки отношения на эквивалентность}

% \textit{Вход.} Матрица $M(\rho)$ бинарного отношения $\rho$ размерности
% $N \times N$.


% \textit{Выход.} <<Отношение является отношением эквивалентности>> или 
% <<Отношение не является отношением эквивалентности>>.

% \begin{enumerate}

%     \item Запустить проверку свойств рефлексивности, транзитивности и симметричности
%     и выполнить операцию \& для их результатов.

%     \item Если получившееся значение истинно, то ответ <<Отношение является
%     отношением эквивалентности>>.

%     \item Если же значение ложно, то ответ <<Отношение не является отношением
%     эквивалентности>>. 

% \end{enumerate}
% Трудоемкость алгоритма $O(N+N^3+N^2) = O(N^3)$ в силу запуска алгоритмов проверки
% свойств рефлексивности, транзитивности и симметричности.

% \subsubsection{Алгоритм проверки отношения на частичный порядок}

% \textit{Вход.} Матрица $M(\rho)$ бинарного отношения $\rho$ размерности
% $N \times N$.

% \textit{Выход.} <<Отношение является
% отношением частичного порядка>> или <<Отношение не является отношением 
% частичного порядка>>.

% \begin{enumerate}
    
%     \item Запустить проверку свойств рефлексивности, транзитивности и 
%     антисимметричности и выполнить операцию \& для их результатов.
    
%     \item Если получившееся значение истинно, то ответ <<Отношение является
%     отношением частичного порядка>>.
    
%     \item Если же получившееся значение ложно, то ответ <<Отношение является
%     отношением частичного порядка>>.

% \end{enumerate}
% Трудоемкость алгоритма $O(N+N^3+N^2) = O(N^3)$ в силу запуска алгоритмов проверки
% свойств рефлексивности, транзитивности и антисимметричности.

% \subsection{Замыкания бинарных отношений и алгоритмы их построения}

% \subsubsection{Определение замыканий отношения}

% \textbf{Замыканием отношения} $R$ относительно свойства $P$ называется такое
% множество $R^*$, что:

% \begin{enumerate}

%     \item $R \subset R^*$.

%     \item $R^*$ Обладает свойством $P$.

%     \item $R^*$ является подмножеством любого другого отношения, содержащего $R$
%     и обладающего свойством $P$. 

%     То есть $R^*$ является минимальным надмножеством множества $R$, 
%     выдерживается $P$.

% \end{enumerate}

% Итак, исходя из вышесказанного, можно сделать вывод, что существуют 4 вида 
% замыканий отношений: \textbf{транзитивное, симметричное, рефлексивное и
% эквивалентное}.

% На множестве $P(A^2)$ всех бинарных отношений между элементами множества $A$ 
% следующие отображения являются операторами замыканий:
% \begin{enumerate}
%     \item $f_r(\rho) = \rho ~\cup \vartriangle_A$ -- наименьшее рефлексивное
%     бинарное отношение, содержащее отношение $\rho \subset A^2$. 
%     \item $f_s(\rho) = \rho \cup \rho^{-1}$ -- наименьшее симметричное
%     бинарное отношение, содержащее отношение $\rho \subset A^2$.
%     \item $f_t(\rho) = \cup^{\infty}_{n=1} \rho^n$ -- наименьшее транзитивное
%     бинарное отношение, содержащее отношение $\rho \subset A^2$.
%     \item $f_{eq}(\rho) = f_tf_sf_r(\rho)$ -- наименьшее отношение эквивалентности,
%     содержащее отношение $\rho \subset A^2$.
    
% \end{enumerate}
% \subsubsection{Пример построения замыканий бинарного отношения}

% Рассмотрим множество $A=$ \{1,2,3,4\}, на котором задано отношение 
% $R=$ {(1,2),(3,4),(4,2)} 

% \begin{enumerate}

%     \item Замыканием $R$ относительно свойства \textbf{рефлексивности}:
%         \begin{center}

%             $R^*=$ \{(1,2),(3,4),(4,2);(1,1),(2,2),(3,3),(4,4)\} 
        
%         \end{center}
  
%     \item Замыканием $R$ относительно свойства \textbf{симметричности}: 
%         \begin{center}
    
%             $R^*=$ \{(1,2),(3,4),(4,2);(2,1),(2,4),(4,3)\} 
    
%         \end{center}
  
%     \item Замыканием $R$ относительно свойства \textbf{транзитивности}: 
%         \begin{center}
        
%             $R^*=$ \{(1,2),(3,4),(4,2);(3,2)\} 
        
%         \end{center}

% \end{enumerate}



% \subsubsection{Построение замыканий отношения}

% Выполним построение замыкания отношения относительно свойства 
% рефлексивности.

% Для этого выполним циклический обход по всем элементам главной диагонали 
% матрицы отношения $M_{ij}$ и будем проверять, находится ли элемент 
% $(M_{ii},M_{ii})$ в исходном отношении.

% \textit{Вход.} Матрица $M(\rho)$ бинарного отношения $\rho$ размерности
% $N \times N$.

% \textit{Выход.} Замыкание относительно свойства рефлексивности.

% \begin{enumerate}
%     \item Создать пустой список для хранения пар замыкания, а также использовать
%     глобальное множество для хранения пар замыкания относительно эквивалентности.
%     \item Цикл по $i ~\text{от}~ 1 ~\text{до}~ N$.
%     \item Если $M_{ii} = 0$, пару $(i, i)$ добавить в замыкание 
%     рефлексивности и замыкание эквивалентности.    
%     \item Ответ - замыкание бинарного отношения $\rho$ относительно рефлексивности.
% \end{enumerate}
% Трудоемкость алгоритма $O(N)$

% Выполним построение замыкания отношения относительно свойства 
% симметричности.

% Для этого выполним циклический обход по всем элементам матрицы 
% отношения $M(\rho)$ и будем проверять, равны ли между собой элементы 
% $M_{ij},M_{ji}$ в исходном отношении.

% \textit{Вход.} Матрица $M(\rho)$ бинарного отношения $\rho$ размерности
% $N \times N$.

% \textit{Выход.} Замыкание относительно свойства симметричности.

% \begin{enumerate}
%     \item Создать пустой список для хранения пар замыкания, а также использовать
%     глобальное множество для хранения пар замыкания относительно эквивалентности.
%     \item Цикл по $i ~\text{от}~ 1 ~\text{до}~ N$, цикл по $j ~\text{от}~ 1 ~\text{до}~ N$.
%     \item Если $M_{ij} = 1 ~\text{и}~ M_{ji} = 0$, добавить пару $(j, i)$
%     в замыкание симметричности и замыкание эквивалентности.
%     \item Ответ - замыкание бинарного отношения $\rho$ относительно симметричности.
% \end{enumerate}
% Трудоемкость алгоритма $O(N^2)$

% Выполним построение замыкания отношения относительно свойства транзитивности.

% Выполним циклический обход по всем элементам матрицы $M_{ij}$ со всеми
% фиксированными элементами $M_{ki}$:

% \begin{enumerate}
%     \item Создать копию матрицы исходного бинарного отношения.
%     \item Цикл по $e ~\text{от}~ 1 ~\text{до}~ N$, цикл по $k ~\text{от}~ 1 ~\text{до}~ N$, 
%     цикл по $i ~\text{от}~ 1 ~\text{до}~ N$, цикл по $j ~\text{от}~ 1 ~\text{до}~ N$.
%     \item Если $M_{ki}=M_{i,j}=1 ~\text{и}~ M_{ki}=0$, то добавить пару
%     $(k, k)$ в замыкание транзитивности и замыкание эквивалентности.
%     \item Ответ - замыкание бинарного отношения $\rho$ относительно транзитивности.
% \end{enumerate}
% Трудоемкость алгоритма $O(N^4)$

% Выполним построение замыкания отношения относительно эквивалентности.

% \begin{enumerate}
%     \item По очереди вызвать алгоритмы построения замыканий относительно рефлексивности,
%     симметричности и транзитивности.
%     \item Ответ - эквивалентное замыкание бинарного отношения $\rho$.
% \end{enumerate}
% Трудоемкость алгоритма $O(N+N^4+N^2) = O(N^4)$ в силу вызова алгоритмов построения
% замыканий относительно рефлексивности, симметричности и транзитивности

% \section{Программная реализация рассмотренных алгоритмов}
    
%     \subsection{Результаты тестирования программы}

%         \begin{figure}[H]
%             \centering
%             \includegraphics[width=0.8\textwidth]{pic/1.jpg}
%             \caption{}
%         \end{figure}
    
%     \subsection{Код программы, реализующей рассмотренные алгоритмы}
    
%         \inputminted[linenos,breaklines=true, fontsize=\small, style=bw]{python}{code/lab1.py}

    
    % \subsection{Оценка сложности реализованных алгоритмов в программе}
    

    % \subsubsection{Алгоритм проверки рефлексивности и антирефлексивности}

    % Как было сказано в теоретической части лабораторной работы, для проверки 
    % отношения на рефлексивность, требуется один проход по главной диагонали
    % матрицы отношения, для чего требуется $O(N)$ операций.

    % \subsubsection{Алгоритмы проверки рефлексивности и антирефлексивности}

    % Как было сказано в теоретической части лабораторной работы, для проверки 
    % отношения на свойства рефлексивности и антирефлексивности,требуется один 
    % проход по главной диагонали матрицы отношения, для чего требуется $O(N)$ 
    % операций.

    % \subsubsection{Алгоритмы проверки симметричности и антисимметричности}
    % Для проверки отношения на свойства симметричности и антисимметричности 
    % в худшем случае требуется полный проход по всем элементам матрицы
    % отношения, что составляет $O(N^2)$ операций.

    % \subsubsection{Алгоритм проверки транзитивности}
    % Для проверки отношения на свойство транзитивности требуется выполнить обход 
    % по всем элементам матрицы отношения (для чего требуется один внешний цикл 
    % и один вложенный) для всех фиксированных элементов из множества $A$.
    % Поэтому асимптотическая оценка данного алгоритма равна $O(N^3)$ операций.

    % \subsubsection{Алгоритм классификации}
    % Время работы алгоритма классификации в худшем случае составляет 
    % $O(N^3 + N^2 + 1)=O(N^3)$.
    

    % \subsubsection{Алгоритм построения рефлексивного замыкания}
    % Время работы алгоритма составляет $O(N)$

    % \subsubsection{Алгоритм построения симметричного замыкания}
    % Время работы алгоритма составляет $O(N^2)$

    % \subsubsection{Алгоритм построения транзитивного замыкания}
    % Время работы алгоритма составляет $O(N^3)$


\conclusion
В ходе лабораторной работы были рассмотрены понятия эквивалентного замыкания
бинарного отношения и получения представителей фактор-множества. Также были
получены алгоритмы вычисления минимальных и максимальных, и наименьших и наибольших
элементов бинарного отношения, а также был определен и программно реализован
алгоритм построения диаграммы Хассе. Был описан алгоритм построения решетки
концептов. Для всех алгоритмов произведена асимптотическая оценка.
\end{document}